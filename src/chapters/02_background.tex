%%%%%%%%%%%%%%%%%%%%
\chapter{Background}
%%%%%%%%%%%%%%%%%%%%

\section{Protocol Learning}

\subsection{Mealy Machines}

\subsection{Equivalence Testing}

\section{Fuzzing}

\subsection{Protocol Fuzzing}

\begin{itemize}
  \item Differences between Protocol Fuzzing and Traditional Fuzzing:
  \begin{enumerate}
    \item Higher complexity in terms of communication: state is building up with every message, both as part of the protocol but more importantly in the implementation.
    \item Need to guarantee additional features that form a more secure or reliable communication: time requirements, authentication, confidentiality, and concurrency
    \item Constrained Testing environment: Testing implementations (especially low-level ones) are tightly coupled to hardware.
    \begin{itemize}
      \item Fuzzing is slower, since there is a dependency on the physical and data link layers
      \item Examples: sectors such as automotive, industrial control system, or baseband processors
    \end{itemize}
  \end{enumerate}
\end{itemize}
\subsection{Architecture of a Protocol Fuzzer}

\begin{itemize}
  \item Refer to \enquote{A survey of Protocol Fuzzing} for a graphical representation
  \item Bug Collection: either via Memory Safety violations (i.e. Sanitization) or semantic bugs (Illegal States)
  \item Execution: on-device Over-The-Air (OTA) or via emulation
  \item Execution Information collection: via rewriting and hooking mechanisms or with externally observable parameters (return values, processing duration, etc.)
\end{itemize}

\subsection{Stateful Programs}

\subsection{Instrumentation}

\begin{itemize}
  \item static instrumentation 
  \item dynamic instrumentation
\end{itemize}


\section{Cellular Networks}

\begin{itemize}
  \item reference: private 5g a systems approach
  \item Explain the architecture of cellular networks (graphically)
  \item Focus on 4G to 5G, how they function, and what their network architecture looks like
  \item Mainly: Centralized authentication and root of trust at Mobile Network Operator (MNO)
\end{itemize}

\subsection{Network Components}


\subsection{Connection Setup Procedure}

\subsection{ASN.1}

\section{Basebands}

\subsection{Baseband Architecture}

\subsubsection{Key Functionalities}
\begin{itemize}
  \item Signal Processing
  \item Protocol Processing
  \item Encryption 
  \item Session Management
  \item Resource Management
\end{itemize}

\subsubsection{Hardware Architecture}
\begin{itemize}
  \item One or more DSP cores - for signal decoding and processing
  \item Real-time-capable MCU core(s) with or without a memory management unit.
  \item Important 
\end{itemize}

\subsubsection{Software Architecture}
\begin{itemize}
  \item Real-Time OS with one task mostly corresponding to one layer of the cellular stack
  \item Mail-box communication system: a task works on a message queue and forwards computed results
  \item Protocol implementations reach from PHY to IP for all supported cellular standards (2G-5G)
  \item Implementations are years old and written / generated mainly in C.
\end{itemize}

For both overviews, provide a practical perspective: what implementations do manufacturers use?
Does a common architecture exist, or are there differences?
\begin{itemize}
  \item Explain that Qualcomm uses Hexagon - DSP architecture even for protocol processing.
  \item New Shannon basebands use an MMU - harder to reverse-engineer/understand/debug.
\end{itemize}


\subsection{Re-Hosting}


\begin{itemize}
  \item Standard Platforms such as Linux/Windows have a standardized I/O interface
  \item Analysis of software running on those platforms is straightforward
  \item Embedded Systems have (undocumented) non-standard I/O-Interfaces
  \item Interaction with several devices, that are not know beforehand but probed during startup/runtime
  \item Provided a memory map with known I/O ports, it is possible to infer/probe the ports and derive a model for the periphery.
  \item Often, when analyzing embedded firmware, it is not of interest to enable all periphery modules, but a subset, relevant to the investigation.
\end{itemize}


\subsection{Security of Cellular Networks}
\subsubsection{User Equipment - System on a Chip}
\begin{itemize}
  \item Application processor
  \item Baseband processor, identification of it via IMEI
  \item SIM for user authentication
\end{itemize}
\subsubsection{Base Stations}
\begin{itemize}
  \item Connection procedure: Network Attachment Sublayer (NAS) with sub-functions
    \begin{itemize}
      \item Mobility Management
      \item Session Management
      \item Connection Management
    \end{itemize}
  \item Registration and Attachment Procedure
  \begin{itemize}
    \item Authentication, and session establishment, release
    \item A high level state machine?
  \end{itemize}
  \item Mobility Management Entity (MME)
  \item After connection success: User Plane Function (UPF)
\end{itemize}
  


\subsection{Open Source eNodeB/gNodeB and UE implementations}

\begin{itemize}
  \item srsRAN: leading and most stable implementation
  \item OpenAirInterface: more feature-rich and experimental implementation
  \item Both exist for 4G and 5G deployments and are set up in this way.
  \item UE implementations: compare implemenation model to real-world implementation in basebands 
\end{itemize}
