%%%%%%%%%%%%%%%%
\chapter{Related Work}
%%%%%%%%%%%%%%%%

The research community eagerly approaches concepts of baseband security analysis.
While initial work focused mainly on manual methods for conducting baseband analysis~\cite{}, the focus has shifted to more automated approaches.

\section{Initial Work}

\begin{itemize}
  \item Attacks on basebands were possible before, but not very economic: SDRs established in the late 2000s
  \item "All your basebands belong to us"~\cite{Weinmann10} constitutes initial work in the area of baseband security research
  \item Vulnerabilities in Apple and Qualcomm modems, that enabled arbitrary code execution on the modems. 
  \item Very critical impact, as it enabled to take phone calls without ringing i.e. record audio remotely.
\end{itemize}

\begin{itemize}
  \item \enquote{A Walk with Shannon}~\cite{Cama18}
  \item Methodology: Read specification and find discrepancy to modem implementation.
  \item Reverse-Engineering of Baseband Modem to verify vulnerability
  \item Collection of RAM-dumps for understanding the memory layout and testing attacks.
\end{itemize}

These results lead to the conclusion that finding vulnerabilities with manual labor led to promising results but requires (i) a deep expertise of basebands and (ii) does not scale well.
This realization has led to increased interest from vulnerability researchers, with a shift towards more complex standards like LTE and 5G, and a focus on a wider variety of manufacturers. 
Additionally, there is a clear need for more scalable approaches.

Current research is exploring three main directions: 
Advanced Static Analysis and Verification, Over-The-Air Fuzzing and, Message Parser Fuzzing via Re-Hosting
We examine each direction below:


\section{Advanced Static Analysis}
\begin{itemize}
  \item BaseSpec by Kim et al. ~\cite{Kim21}:
    \begin{itemize}
      \item Compares protocol specifications with actual implementations
      \item Extracts data structures in both
      \item and finally compares them.
      \item Discrepancies are analyzed and classified as vulnerabilities.
    \end{itemize}
  \item Building on this, BaseComp was introduced as the successor work~\cite{Kim23}
    \begin{itemize}
      \item Aims to compare specifically the integrity protection functions in cellular basebands
      \item Reduces manual efforts
      \item Compares Control Flow Graphs (CFGs) from baseband firmware with specifications
      \item Limitation: Only one subcomponent of the stack is analyzed
    \end{itemize}
  \item Liu et al~\cite{Liu24} propose a method for analyzing baseband firmware statically.
    \begin{itemize}
      \item Identify memory accesses with attacker-controllable input
      \item Taint vulnerable programming patterns 
      \item Propagate the tainted positions to entry points
    \end{itemize}
\end{itemize}

These approaches share common techniques:
\begin{itemize}
  \item They are specification-based
  \item They use symbolic execution and static analysis
  \item They extract data structures and CFGs form the modems
\end{itemize}

Corresponding to the following advantages:
\begin{itemize}
  \item Finding logical errors is easily possible as the specification is taken as a reference
\end{itemize}
However, they also face significant challenges:
\begin{itemize}
  \item Time-consuming and incomplete analysis
  \item Consequence: Scalability remains an issue
  \item They can produce a high number of false positives
\end{itemize}

\section{Over-The-Air Fuzzing}

\section{Message Parser Fuzzing via Re-Hosting}


The key motivation for this re-hosting based approach is to maintain the true positives from OTA testing while gaining visibility into the execution trace.
Emulation provides this crucial insight into coverage information.

Two particularly impactful approaches have emerged from the field of embedded device re-hosting:
\begin{enumerate}
  \item BaseSAFE by Maier et al.~\cite{Maier2020}
    \begin{itemize}
      \item Selectively re-hosts parts of the modem, focusing on parser vulnerabilities
      \item adds hooks to memory management functions and injects sanitization behavior
      \item allows for coverage-guided fuzzing and bug identification
      \item Limitation: due to its selective nature, it does not achieve a 1:1 behavior with a real modem, potentially leading to false positives
    \end{itemize}
  \item Firmwire by Hernandez et al.~\cite{Hernandez2022}:
    \begin{itemize}
      \item Approach: Re-hosting the entire device and (a subset of) its peripherals
      \item Extensibility: framework to simplify re-hosting of similar modems
      \item Working Principle: emulates peripherals outside the device, giving the illusion of a real modem environment.
      \item Limitation: fuzzing focuses on modem tasks, particularly targeting parser vulnerabilities through mutational fuzzing
    \end{itemize}
\end{enumerate}

The advantages of these emulation-based approaches are significant:
\begin{itemize}
  \item No need for hardware-in-the-loop testing
  \item Dynamic rewriting allows for additional features such as sanitization
  \item Enables crucial insight into code coverage
\end{itemize}

However, they also face challenges:
\begin{itemize}
  \item High complexity in implementation
  \item Difficulties in accurate periphery modeling
  \item Limited coverage due to reliance on mutational fuzzing
\end{itemize}