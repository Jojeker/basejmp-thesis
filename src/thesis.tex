%%%%%%%%%%%%%%%%%%%%%%%%%%%%%%%%%%%%%%%%%%%%%%%%%%%%%%%%%%%
% EPFL report package, main thesis file
% Goal: provide formatting for theses and project reports
% Author: Mathias Payer <mathias.payer@epfl.ch>
%
% To avoid any implication, this template is released into the
% public domain / CC0, whatever is most convenient for the author
% using this template.
%
%%%%%%%%%%%%%%%%%%%%%%%%%%%%%%%%%%%%%%%%%%%%%%%%%%%%%%%%%%%
\documentclass[a4paper,11pt,oneside]{report}
% Options: MScThesis, BScThesis, MScProject, BScProject
\usepackage[MScThesis,lablogo]{EPFLreport}
\usepackage{xspace}

\title{A wonderful thesis\\about the merits of scientific writing}
\author{The Student}
\supervisor{The Doctoral Student}
\adviser{Prof. Dr. sc. ETH Mathias Payer}
%\coadviser{Second Adviser}
\expert{The External Reviewer}

\newcommand{\sysname}{FooSystem\xspace}

\begin{document}
\maketitle
\makededication
\makeacks

\begin{abstract}
To be commpleted
\end{abstract}

\begin{frenchabstract}
To be completed
\end{frenchabstract}

\maketoc

%%%%%%%%%%%%%%%%%%%%%%
\chapter{Introduction}
%%%%%%%%%%%%%%%%%%%%%%

This section is usually 3--5 pages.

%%%%%%%%%%%%%%%%%%%%
\chapter{Background}
%%%%%%%%%%%%%%%%%%%%

\section{Cellular Networks}
\begin{itemize}
  \item Explain the architecture of cellular networks (graphically)
  \item Focus on 4G to 5G, how they function, and what their network architecture looks like
  \item Mainly: Centralized authentication and root of trust at Mobile Network Operator (MNO)
\end{itemize}

\subsection{Security of Cellular Networks}
\subsubsection{User Equipment - System on a Chip}
\begin{itemize}
  \item Application processor
  \item Baseband processor, identification of it via IMEI
  \item SIM for user authentication
\end{itemize}
\subsubsection{Base Stations}
\begin{itemize}
  \item Connection procedure: Network Attachment Sublayer (NAS) with sub-functions
    \begin{itemize}
      \item Mobility Management
      \item Session Management
      \item Connection Management
    \end{itemize}
  \item Registration and Attachment Procedure
  \begin{itemize}
    \item Authentication, and session establishment, release
    \item A high level state machine?
  \end{itemize}
  \item Mobility Management Entity (MME)
  \item After connection success: User Plane Function (UPF)
\end{itemize}
  
\subsection{Baseband Processors}

\subsubsection{Key Functionalities}
\begin{itemize}
  \item Signal Processing
  \item Protocol Processing
  \item Encryption 
  \item Session Management
  \item Resource Management
\end{itemize}

\subsubsection{Hardware Architecture}
\begin{itemize}
  \item One or more DSP cores - for signal decoding and processing
  \item Real-time-capable MCU core(s) with or without a memory management unit.
  \item Important 
\end{itemize}

\subsubsection{Software Architecture}
\begin{itemize}
  \item Real-Time OS with one task mostly corresponding to one layer of the cellular stack
  \item Mail-box communication system: a task works on a message queue and forwards computed results
  \item Protocol implementations reach from PHY to IP for all supported cellular standards (2G-5G)
  \item Implementations are years old and written / generated mainly in C.
\end{itemize}

For both overviews, provide a practical perspective: what implementations do manufacturers use?
Does a common architecture exist, or are there differences?
\begin{itemize}
  \item Explain that Qualcomm uses Hexagon - DSP architecture even for protocol processing.
  \item New Shannon basebands use an MMU - harder to reverse-engineer/understand/debug.
\end{itemize}


\subsection{Open Source eNodeB/gNodeB and UE implementations}

\begin{itemize}
  \item srsRAN: leading and most stable implementation
  \item OpenAirInterface: more feature-rich and experimental implementation
  \item Both exist for 4G and 5G deployments and are set up in this way.
  \item UE implementations: compare implemenation model to real-world implementation in basebands 
\end{itemize}


\section{Protocol Fuzzing}

\begin{itemize}
  \item Differences between Protocol Fuzzing and Traditional Fuzzing:
  \begin{enumerate}
    \item Higher complexity in terms of communication: state is building up with every message, both as part of the protocol but more importantly in the implementation.
    \item Need to guarantee additional features that form a more secure or reliable communication: time requirements, authentication, confidentiality, and concurrency
    \item Constrained Testing environment: Testing implementations (especially low-level ones) are tightly coupled to hardware.
    \begin{itemize}
      \item Fuzzing is slower, since there is a dependency on the physical and data link layers
      \item Examples: sectors such as automotive, industrial control system, or baseband processors
    \end{itemize}
  \end{enumerate}
\end{itemize}

\subsection{Architecture of a Protocol Fuzzer}
\begin{itemize}
  \item Refer to \enquote{A survey of Protocol Fuzzing} for a graphical representation
  \item Bug Collection: either via Memory Safety violations (i.e. Sanitization) or semantic bugs (Illegal States)
  \item Execution: on-device Over-The-Air (OTA) or via emulation
  \item Execution Information collection: via rewriting and hooking mechanisms or with externally observable parameters (return values, processing duration, etc.)
\end{itemize}

\subsection{Re-Hosting}
\begin{itemize}
  \item Standard Platforms such as Linux/Windows have a standardized I/O interface
  \item Analysis of software running on those platforms is straightforward
  \item Embedded Systems have (undocumented) non-standard I/O-Interfaces
  \item Interaction with several devices, that are not know beforehand but probed during startup/runtime
  \item Provided a memory map with known I/O ports, it is possible to infer/probe the ports and derive a model for the periphery.
  \item Often, when analyzing embedded firmware, it is not of interest to enable all periphery modules, but a subset, relevant to the investigation.
\end{itemize}

%%%%%%%%%%%%%%%%
\chapter{Related Work}
%%%%%%%%%%%%%%%%

The research community eagerly approaches concepts of baseband security analysis.
While initial work focused mainly on manual methods for conducting baseband analysis~\cite{}, the focus has shifted to more automated approaches.

\section{Initial Work}

\begin{itemize}
  \item Attacks on basebands were possible before, but not very economic: SDRs established in the late 2000s
  \item "All your basebands belong to us"~\cite{Weinmann10} constitutes initial work in the area of baseband security research
  \item Vulnerabilities in Apple and Qualcomm modems, that enabled arbitrary code execution on the modems. 
  \item Very critical impact, as it enabled to take phone calls without ringing i.e. record audio remotely.
\end{itemize}

\begin{itemize}
  \item \enquote{A Walk with Shannon}~\cite{Cama18}
  \item Methodology: Read specification and find discrepancy to modem implementation.
  \item Reverse-Engineering of Baseband Modem to verify vulnerability
  \item Collection of RAM-dumps for understanding the memory layout and testing attacks.
\end{itemize}

These results lead to the conclusion that finding vulnerabilities with manual labor led to promising results but requires (i) a deep expertise of basebands and (ii) does not scale well.
This realization has led to increased interest from vulnerability researchers, with a shift towards more complex standards like LTE and 5G, and a focus on a wider variety of manufacturers. 
Additionally, there is a clear need for more scalable approaches.

Current research is exploring three main directions: 
Advanced Static Analysis and Verification, Over-The-Air Fuzzing and, Message Parser Fuzzing via Re-Hosting
We examine each direction below:


\section{Advanced Static Analysis}
\begin{itemize}
  \item BaseSpec by Kim et al. ~\cite{Kim21}:
    \begin{itemize}
      \item Compares protocol specifications with actual implementations
      \item Extracts data structures in both
      \item and finally compares them.
      \item Discrepancies are analyzed and classified as vulnerabilities.
    \end{itemize}
  \item Building on this, BaseComp was introduced as the successor work~\cite{Kim23}
    \begin{itemize}
      \item Aims to compare specifically the integrity protection functions in cellular basebands
      \item Reduces manual efforts
      \item Compares Control Flow Graphs (CFGs) from baseband firmware with specifications
      \item Limitation: Only one subcomponent of the stack is analyzed
    \end{itemize}
  \item Liu et al~\cite{Liu24} propose a method for analyzing baseband firmware statically.
    \begin{itemize}
      \item Identify memory accesses with attacker-controllable input
      \item Taint vulnerable programming patterns 
      \item Propagate the tainted positions to entry points
    \end{itemize}
\end{itemize}

These approaches share common techniques:
\begin{itemize}
  \item They are specification-based
  \item They use symbolic execution and static analysis
  \item They extract data structures and CFGs form the modems
\end{itemize}

Corresponding to the following advantages:
\begin{itemize}
  \item Finding logical errors is easily possible as the specification is taken as a reference
\end{itemize}
However, they also face significant challenges:
\begin{itemize}
  \item Time-consuming and incomplete analysis
  \item Consequence: Scalability remains an issue
  \item They can produce a high number of false positives
\end{itemize}

\section{Over-The-Air Fuzzing}

\section{Message Parser Fuzzing via Re-Hosting}


The key motivation for this re-hosting based approach is to maintain the true positives from OTA testing while gaining visibility into the execution trace.
Emulation provides this crucial insight into coverage information.

Two particularly impactful approaches have emerged from the field of embedded device re-hosting:
\begin{enumerate}
  \item BaseSAFE by Maier et al.~\cite{Maier2020}
    \begin{itemize}
      \item Selectively re-hosts parts of the modem, focusing on parser vulnerabilities
      \item adds hooks to memory management functions and injects sanitization behavior
      \item allows for coverage-guided fuzzing and bug identification
      \item Limitation: due to its selective nature, it does not achieve a 1:1 behavior with a real modem, potentially leading to false positives
    \end{itemize}
  \item Firmwire by Hernandez et al.~\cite{Hernandez2022}:
    \begin{itemize}
      \item Approach: Re-hosting the entire device and (a subset of) its peripherals
      \item Extensibility: framework to simplify re-hosting of similar modems
      \item Working Principle: emulates peripherals outside the device, giving the illusion of a real modem environment.
      \item Limitation: fuzzing focuses on modem tasks, particularly targeting parser vulnerabilities through mutational fuzzing
    \end{itemize}
\end{enumerate}

The advantages of these emulation-based approaches are significant:
\begin{itemize}
  \item No need for hardware-in-the-loop testing
  \item Dynamic rewriting allows for additional features such as sanitization
  \item Enables crucial insight into code coverage
\end{itemize}

However, they also face challenges:
\begin{itemize}
  \item High complexity in implementation
  \item Difficulties in accurate periphery modeling
  \item Limited coverage due to reliance on mutational fuzzing
\end{itemize}

%%%%%%%%%%%%%%%%
\chapter{Design}
%%%%%%%%%%%%%%%%

Introduce and discuss the design decisions that you made during this project.
Highlight why individual decisions are important and/or necessary. Discuss
how the design fits together.

This section is usually 5-10 pages.


%%%%%%%%%%%%%%%%%%%%%%%%
\chapter{Implementation}
%%%%%%%%%%%%%%%%%%%%%%%%

The implementation covers some of the implementation details of your project.
This is not intended to be a low level description of every line of code that
you wrote but covers the implementation aspects of the projects.

This section is usually 3-5 pages.


%%%%%%%%%%%%%%%%%%%%
\chapter{Evaluation}
%%%%%%%%%%%%%%%%%%%%

In the evaluation you convince the reader that your design works as intended.
Describe the evaluation setup, the designed experiments, and how the
experiments showcase the individual points you want to prove.

This section is usually 5-10 pages.


%%%%%%%%%%%%%%%%%%%%%%
\chapter{Related Work}
%%%%%%%%%%%%%%%%%%%%%%

The related work section covers closely related work. Here you can highlight
the related work, how it solved the problem, and why it solved a different
problem. Do not play down the importance of related work, all of these
systems have been published and evaluated! Say what is different and how
you overcome some of the weaknesses of related work by discussing the 
trade-offs. Stay positive!

This section is usually 3-5 pages.


%%%%%%%%%%%%%%%%%%%%
\chapter{Conclusion}
%%%%%%%%%%%%%%%%%%%%

In the conclusion you repeat the main result and finalize the discussion of
your project. Mention the core results and why as well as how your system
advances the status quo.

\cleardoublepage
\phantomsection
\addcontentsline{toc}{chapter}{Bibliography}
\printbibliography

% Appendices are optional
% \appendix
% %%%%%%%%%%%%%%%%%%%%%%%%%%%%%%%%%%%%%%
% \chapter{How to make a transmogrifier}
% %%%%%%%%%%%%%%%%%%%%%%%%%%%%%%%%%%%%%%
%
% In case you ever need an (optional) appendix.
%
% You need the following items:
% \begin{itemize}
% \item A box
% \item Crayons
% \item A self-aware 5-year old
% \end{itemize}

\end{document}
